\section{M : Files}
\label{chap:files}

%%%%%%%%%%%%%%%%%%%%%%%%%%%%%%%%%%%%%%%%%%%%%%%%%%%%%%%%%%%%%%

\begin{frame}[fragile]
\frametitle{File Properties}
\begin{columns}[T]


\begin{column}{0.55\textwidth}
\begin{itemize}[<+->]
\item They have a name.
\item Until a file is opened nothing can be done with it.
\item After use a file must be closed.
\item Files may be read, written or appended.
\item Conceptually a file may be thought of as a stream of characters.
\end{itemize}
\end{column}

\pause
\begin{column}{0.35\textwidth}
\lstinputlisting[style=basicc]{../Code/ChapM/fhello.c}
\end{column}

\end{columns}
\end{frame}

%%%%%%%%%%%%%%%%%%%%%%%%%%%%%%%%%%%%%%%%%%%%%%%%%%%%%%%%%%%%%%

\begin{frame}[fragile]
\frametitle{Reading and Writing}
\begin{columns}[T]

\begin{column}{0.50\textwidth}
\lstinputlisting[style=basicc]{../Code/ChapM/readwrite.c}
\end{column}

\pause
\begin{column}{0.40\textwidth}
\begin{itemize}[<+->]
\item If you write a file, it overwrites it from the beginning.
\item You must \verb^fclose()^ your file pointers otherwise there is a memory leak.
\item The statement \verb^exit()^ allows you to exit the code {\bf anywhere}, not just in \verb^main^.
\item There are three files already open for you: \verb^stdin^, \verb^stdout^ and \verb^stderr^.
\item Therefore \verb^printf(...)^ is just a shorthand for \verb^fprintf(stdout, ...)^
\item \verb^fscanf()^ could be used instead of \verb^fgets()^.
\end{itemize}
\end{column}


\end{columns}
\end{frame}

%%%%%%%%%%%%%%%%%%%%%%%%%%%%%%%%%%%%%%%%%%%%%%%%%%%%%%%%%%%%%%

\begin{frame}[fragile]
\frametitle{{\em stderr}}
\begin{columns}[T]

\begin{column}{0.90\textwidth}
\begin{itemize}[<+->]
\item To write to screen you'd generally use \verb^stdout^, so why is there \verb^stderr^ ?
\item It's fairly common, when running a program, to want to redirect the output to a file.
\item For instance, if you were to type:
\begin{verbatim}
$ ls > myfiles.txt
\end{verbatim}
this will list all your files into myfiles.txt
\item The ">" at the prompt redirects the output (or "<" input) to file rather than screen.
\item If something went wrong though, the user would never see the message.
\item Therefore, \verb^stderr^ exists so that there is a stream to display warnings/errors to the user.
\end{itemize}
\end{column}

\end{columns}
\end{frame}

%%%%%%%%%%%%%%%%%%%%%%%%%%%%%%%%%%%%%%%%%%%%%%%%%%%%%%%%%%%%%%

\begin{frame}[fragile]
\frametitle{One Character at a Time}
\begin{columns}[T]

\begin{column}{0.50\textwidth}
\lstinputlisting[style=basicc]{../Code/ChapM/mycp.c}
\end{column}

\pause
\begin{column}{0.40\textwidth}
\begin{itemize}[<+->]
\item This is a very basic version of the Linux command \verb^cp^.
\item \verb^fgetc()^ and \verb^fputc()^ are the file equivalents of \verb^getchar^ and \verb^putchar^.
\end{itemize}
\end{column}


\end{columns}
\end{frame}

%%%%%%%%%%%%%%%%%%%%%%%%%%%%%%%%%%%%%%%%%%%%%%%%%%%%%%%%%%%%%%

\begin{frame}[fragile]
\frametitle{Bulk Copying}
\begin{columns}[T]

\begin{column}{0.40\textwidth}
\begin{itemize}[<+->]
\item Copying one character at a time is very slow for large files.
\item \verb^fread()^ and \verb^fwrite()^ will I/O many characters at once.
\item Here we save an entire array to a binary file - another program could
read this in later.
\end{itemize}
\end{column}

\pause
\begin{column}{0.50\textwidth}
\lstinputlisting[style=basicc]{../Code/ChapM/bulkfacts.c}
\end{column}


\end{columns}
\end{frame}

%%%%%%%%%%%%%%%%%%%%%%%%%%%%%%%%%%%%%%%%%%%%%%%%%%%%%%%%%%%%%%

\begin{frame}[fragile]
\frametitle{Window/DOS vs. Unix Text Files}
\begin{columns}[T]


\begin{column}{0.45\textwidth}
\begin{itemize}[<+->]
\item Text files created on DOS/Windows machines have different line endings than files created on Unix/Linux.
\item DOS uses {\em carriage return} and {\em line feed} \verb^("\r\n")^ as a line ending.
\item Unix uses just {\em line feed} \verb^("\n")^.
\item You should generally avoid assuming line endings - use a function that reads entire lines at once such as \verb^fgets()^.
\item When you open a file in textmode \verb^fopen("file.txt", "rt")^ some automic translation may be done on input/output.
\end{itemize}
\end{column}

\pause
\begin{column}{0.45\textwidth}
\lstinputlisting[style=basicc]{../Code/ChapM/unixordos.c}
\end{column}

\end{columns}
\end{frame}


\endinput

